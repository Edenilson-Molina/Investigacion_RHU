\documentclass[journal]{IEEEtran}
\usepackage[utf8]{inputenc}
\usepackage[T1]{fontenc}
\usepackage[spanish]{babel}
\usepackage{graphicx}
\usepackage{url}
\usepackage{hyperref}
\usepackage{float}
\hyphenation{op-tical net-works semi-conduc-tor}

\begin{document}

% paper title
\title{Dirección Efectiva en la Ingeniería de Sistemas Informáticos: Un Análisis de la Inteligencia Emocional y las Técnicas de Relaciones Humanas}

\author{Elmer Edenilson Rosales Molina,\\ ~\IEEEmembership{rm20001@ues.edu.sv;}\\
        Julio Cesar Mejía Rodriguez,\\ ~\IEEEmembership{mr20084@ues.edu.sv}}

\maketitle

\begin{abstract}
The abstract goes here.
\end{abstract}

\begin{keywords}
IEEEtran, journal, \LaTeX, paper, template.
\end{keywords}

\IEEEpeerreviewmaketitle

\section{Introducción}

 
\hfill Febrero 02, 2024

\section{Objetivos}

\section{Alcances}

\section{Organización y Relaciones Interpersonales}
\PARstart{E}{n} el mundo de la dirección, especialmente en el campo de la Ingeniería de Sistemas Informáticos, la organización y las relaciones interpersonales asumen un papel crucial. Debido a que la organización hace referencia a cómo se estructuran y coordinan las actividades y los recursos dentro de una empresa (lo cual no es tarea fácil), por consiguiente, las relaciones interpersonales se refieren a cómo los miembros del equipo interactúan entre sí para obtener metas establecidas.

En su libro “Cómo ganar amigos e influir sobre las personas”, Dale Carnegie presenta varias técnicas que pueden mejorar las relaciones interpersonales en un entorno de trabajo. Por ejemplo, sugiere que mostrar un interés genuino en los demás y ser un buen oyente puede ayudar a construir relaciones más fuertes y más positivas. Estas técnicas pueden ser especialmente útiles en un entorno de TI, donde la colaboración y la comunicación efectiva son esenciales para el éxito del proyecto.

Además, Carnegie enfatiza la importancia de hacer que las personas se sientan importantes y valoradas. Esta técnica puede ser especialmente relevante para la organización en un entorno de TI. Al hacer que los miembros del equipo se sientan valorados, los líderes pueden aumentar la motivación y el compromiso del equipo, lo cual puede llevar a una mayor productividad y eficacia.

Por otro lado, en su libro “Inteligencia Emocional”, Daniel Goleman argumenta que la capacidad para reconocer, entender y gestionar nuestras emociones y las de los demás es crucial para las relaciones interpersonales. En un entorno de TI, donde los proyectos a menudo pueden ser estresantes y desafiantes, la inteligencia emocional puede ayudar a los líderes a manejar mejor el estrés y a resolver conflictos de manera efectiva

\subsection{Subsection Heading Here}
Subsection text here.

\subsubsection{Subsubsection Heading Here}
Subsubsection text here.


%\begin{table}[H]
%% increase table row spacing, adjust to taste
%\renewcommand{\arraystretch}{1.3}
%\caption{An Example of a Table}
%\label{table_example}
%\centering
%\begin{tabular}{|c||c|}
%\hline
%One & Two\\
%\hline
%Three & Four\\
%\hline
%\end{tabular}
%\end{table}


\section{Comunicación Efectiva en la Dirección de Proyectos de TI}
La comunicación efectiva se considera una habilidad esencial para el éxito de cualquier proyecto, debido a que todos los participantes muestran lo mejor de sí mismos y actuan con sinergia en todas las actvidades del proyecto al crear un clima de trabajo agradable. 

Al hacer referencia en el campo de la Ingeniería de Sistemas Informáticos los proyectos de TI suelen ser complejos y requieren de un equipo compuesto por individuos con diferentes habilidades y conocimientos. Usualmente estos son multidisciplinarios y dinámicos, se necesita interactuar constantemente con diferentes actores.

Según Chavez, “Una buena comunicación permite establecer objetivos claros, alinear expectativas, resolver conflictos, compartir información, obtener feedback, generar confianza y fomentar la colaboración. Todo esto contribuye a mejorar la calidad, la eficiencia y la satisfacción de los proyectos de TI”.

\subsection{Aspectos que deben considerarse en cada proyecto}
En proyectos informáticos, la calidad del producto se debe múltiples factores que pueden ser cumplidos al trabajar en conjunto y de forma coordinada, por esta razón, el dirigente del proyecto debe considerar los siguientes puntos:
\begin{itemize}
	\item Definir el propósito y el alcance del proyecto, así como los roles y responsabilidades de cada participante.
	\item Utilizar los canales y herramientas adecuados para cada tipo de comunicación, ya sea oral, escrita, sincrónica o asincrónica.
	\item Ser claro, conciso, coherente y cortés en tus mensajes, evitando ambigüedades, jergas y tecnicismos innecesarios.
	\item Escuchar activamente a los demás, mostrando empatía, respeto y apertura.
	\item Solicitar y ofrece feedback constructivo, reconociendo los logros y las oportunidades de mejora.
	\item Mantener una comunicación fluida y frecuente, informando sobre el estado, los avances y los problemas del proyecto.
	\item Fomentar la participación y la colaboración de todos los involucrados, creando un ambiente de confianza y transparencia.
\end{itemize}

\subsection{El enfoque de Carnegie para mejorar la comunicación}
Daniel Carnegie en su libro “Cómo ganar amigos e influir sobre las personas” proporciona técnicas para interactuar eficazmente con las personas lo cual contribuye a la buena comunicación entre los miembros del equipo. Por lo tanto, se citarán algunas de estas tecnicas:

\subsubsection{No critique, no condene ni se queje}
Según Carnegie, “La crítica es inútil porque pone a la otra persona en la defensiva, y por lo común hace que trate de justificarse. La crítica es peligrosa porque lastima el orgullo, tan precioso de la persona, hiere su sentido de la importancia y despierta su resentimiento.”

Ahora bien, de acuerdo a lo expuesto por este autor es posible que el líder de un proyecto se sienta tentado ha actuar con esa conducta lo cual no ayudaría en mucho. Si se analiza detenidamente en escenas de trabajo, podemos obtener lo siguiente:

\begin{itemize}
	\item \textbf{No critique:} En la dirección de proyectos de TI, es común encontrar errores y problemas. Sin embargo, criticar a los miembros del equipo por sus errores puede crear un ambiente de trabajo negativo y disminuir la moral del equipo. En lugar de criticar, un líder efectivo proporciona retroalimentación constructiva y ayuda a los miembros del equipo a aprender de sus errores.
	\item \textbf{No condene:} Condenar a los miembros del equipo por sus acciones puede crear resentimiento y conflictos dentro del equipo. Un líder efectivo se enfoca en resolver el problema en lugar de condenar a la persona. Esto puede incluir discutir el problema de manera abierta y honesta, buscar soluciones juntos y aprender de la experiencia.
	\item \textbf{No se queje: }Quejarse puede ser perjudicial para el ambiente de trabajo y puede disminuir la productividad del equipo. En lugar de quejarse, un líder efectivo se enfoca en encontrar soluciones a los problemas y mantener una actitud positiva.
\end{itemize}

Por lo expuesto anteriormente, metodologías agiles como SCRUM ofrecen espacios entre los miembros para exponer los avances realizados, además, cuando uno de ellos presenta dificultades para desempeñar su tarea todos los miembros del equipo unen su conocimiento y experiencia para solucionar el problema lo cual hace pensar la importancia de la comunicación.

\subsubsection{Demuestre aprecio honrado y sincero}
Se cita un viejo dicho, “Pasaré una sola vez por este camino; de modo que cualquier bien que pueda hacer o cualquier cortesía que pueda tener para con cualquier ser humano, que sea ahora. No la dejaré para mañana, ni la olvidaré, porque nunca más volveré a pasar por aquí.”

Según Carnegie, “Herir a la gente no sólo no la cambia, sino que es una tarea que nadie nos agradecerá”. En esta regla señala que el caracter humano posee una principio profundo que es el anhelo de ser apreciado, eso emmarca algo significativo en las personas y satisfacen sus deseos lo cual es muy diferente a la adulación porque genera sentido de importancia y mejora la calidad.

En ingeniería se puede reflexionar en la siguiente aplicación:
\begin{itemize}
	\item \textbf{Reconocimiento del trabajo bien hecho:} En la dirección de proyectos de TI, es importante reconocer y apreciar el trabajo bien hecho. Esto puede ser tan simple como agradecer a un miembro del equipo por su contribución o destacar el buen trabajo en una reunión de equipo. Este tipo de aprecio sincero puede aumentar la moral del equipo y motivar a los miembros del equipo a continuar haciendo un buen trabajo.
	\item \textbf{Valoración de las ideas y sugerencias:} Los líderes de proyectos de TI deben demostrar aprecio por las ideas y sugerencias de los miembros del equipo. Esto no solo fomenta un ambiente de innovación y creatividad, sino que también hace que los miembros del equipo se sientan valorados y respetados.
	\item \textbf{Apoyo durante los desafíos:} Los proyectos de TI a menudo pueden ser desafiantes y estresantes. Durante estos tiempos, demostrar aprecio por el esfuerzo y la dedicación de los miembros del equipo puede proporcionar un muy necesario impulso moral.
\end{itemize}

\subsubsection{Despierte en los demás un deseo vehemente}
En la dirección de proyectos de TI, es importante inspirar a los miembros del equipo a alcanzar sus metas y objetivos. Según Carnegie, “el único medio que se tiene para influir sobre la otra persona es hablar acerca de lo que se quiere y demostrarle cómo conseguirlo”. Esto puede implicar ayudar a los miembros del equipo a ver cómo su trabajo contribuye al éxito general del proyecto y cómo pueden crecer y desarrollarse profesionalmente.

Además, despertar un deseo vehemente en los demás también puede ser una poderosa herramienta de motivación. Carnegie argumenta que “quien puede hacer las cosas tiene al mundo entero consigo. Quien no puede, marcha solo por el camino”. En un entorno de TI, esto podría implicar motivar a los miembros del equipo a aprender nuevas habilidades, adoptar nuevas tecnologías o asumir nuevos desafíos.

Al despertar un deseo vehemente en los demás, los líderes de proyectos de TI pueden influir positivamente en sus equipos. Esto puede llevar a un mayor compromiso, una mayor productividad y, en última instancia, a un mayor éxito del proyecto.

\subsubsection{Seis maneras de agradar a los demás}
\begin{itemize}
	\item \textbf{Interésese sinceramente por los demás:} Carnegie sugiere mostrar un interés genuino en los demás. En la dirección de proyectos de TI, esto puede traducirse en prestar atención a las ideas y opiniones de los miembros del equipo, lo cual puede fomentar un ambiente de trabajo colaborativo y respetuoso.
	\item \textbf{Sonría:} Una actitud positiva puede ser contagiosa y puede ayudar a crear un ambiente de trabajo más agradable y productivo.
	\item \textbf{Recuerde que para toda persona su nombre es el sonido más dulce e importante en cualquier idioma:} Recordar y usar los nombres de los miembros del equipo puede demostrar respeto y aprecio, lo cual puede mejorar las relaciones interpersonales y la comunicación.
	\item \textbf{Sea un buen oyente. Anime a los demás a hablar de sí mismos:} Escuchar activamente a los miembros del equipo puede demostrar que valoras sus ideas y opiniones, lo cual puede fomentar una comunicación abierta y efectiva.
	\item \textbf{Hable en términos de los intereses de la otra persona:} Comprender y hablar en términos de los intereses de los miembros del equipo puede ayudar a motivarlos y a obtener su cooperación.
	\item \textbf{Haga que la otra persona se sienta importante y hágalo sinceramente:} Reconocer el valor y las contribuciones de los miembros del equipo puede aumentar su moral y su compromiso con el proyecto.
\end{itemize}

\section{El Liderazgo}
En la dirección de proyectos de TI, es común encontrarse con errores y problemas complejos. Sin embargo, la forma en que un líder maneja estos problemas puede tener un gran impacto en el equipo y en el éxito del proyecto. Dale Carnegie ofrece varias técnicas que pueden ayudar a los líderes a manejar estos problemas de manera efectiva.

\subsection{Enfoque de Dale Carnegie}

\subsubsection{Elogios y aprecio sincero}
Para abordar un problema o un error primeramente Carnegie sugiere comenzar con elogios y aprecio sincero con el fin de obtener respuestas positivas. Esto puede ayudar a suavizar el golpe de la crítica y a hacer que la otra persona esté más abierta a la retroalimentación.

\subsubsection{Llamar la atención indirectamente}
En lugar de señalar directamente los errores de los demás, Carnegie sugiere hacerlo de manera indirecta. Esto puede ser especialmente útil en un entorno de TI, donde la crítica directa puede ser mal recibida y causar resentimiento.
 
\subsubsection{Hablar primero de nuestros errores antes de críticar a los demás}
Admitir nuestros propios errores antes de señalar los de los demás puede hacer que la crítica sea más fácil de aceptar. Esto puede ser especialmente relevante en un entorno de TI, donde los errores son a menudo inevitables y la mejora continua es clave.
 
\subsubsection{Hacer preguntas en lugar de dar órdenes directas}
Carnegie sugiere que hacer preguntas puede ser una forma efectiva de influir en los demás sin causar resentimiento. En un entorno de TI, esto podría implicar pedir a los miembros del equipo su opinión o sugerencias sobre cómo resolver un problema en lugar de simplemente decirles qué hacer.
 
\subsection{Enfoque de Daniel Goleman}
 
\subsubsection{Autoconsciencia:} Goleman argumenta que la autoconsciencia, la capacidad de reconocer y entender nuestras propias emociones, es la base de la inteligencia emocional. En la dirección de proyectos de TI, la autoconsciencia puede ayudar a los líderes a entender cómo sus emociones pueden influir en su comportamiento y en sus decisiones.
\subsubsection{Autorregulación:} La autorregulación, la capacidad de manejar nuestras emociones de manera efectiva, es otro aspecto clave de la inteligencia emocional. Los líderes que pueden manejar sus emociones son menos propensos a tomar decisiones impulsivas o a reaccionar de manera exagerada ante los problemas.
\subsubsection{Empatía:} La empatía, la capacidad de entender y compartir los sentimientos de los demás, es crucial para el liderazgo efectivo. Los líderes empáticos pueden entender mejor las necesidades y preocupaciones de sus equipos, lo cual puede mejorar la comunicación y la colaboración.
\subsubsection{Habilidades sociales:} Goleman también destaca la importancia de las habilidades sociales, como la capacidad de manejar las relaciones y navegar por las redes sociales. En un entorno de TI, donde la colaboración es clave, las habilidades sociales pueden ayudar a los líderes a construir relaciones fuertes y productivas con sus equipos.

\section{La Cultura Organizacional y Ambiente de Trabajo}
Son aspectos fundamentales dentro de una organización porque expresan lo bien de una entidad con el trato al personal de trabajo, y por ende, las buenas ideas le dan lugar al negocio.

\subsection{Cultura organizacional}
La cultura organizacional es el conjunto de valores, creencias, normas, costumbres y comportamientos compartidos por los miembros de una organización.

Define la identidad de la empresa y afecta cómo las personas interactuan entre sí y con la organización en su conjunto. Influye en la toma de decisiones, la comunicación, la motivación y la forma en que abordan los desafíos.

Dale Carnegie, en su libro "Cómo ganar amigos e influir sobre las personas" se centra exclusivamente en las habilidades interpersonales y de comunicación, pues está interesado en mostrar una perspectiva motivacional en la interacción con las personas. De acuerdo al autor podemos mostrar dos aspectos relevantes ha considerar para la cultura organizacional:

\subsubsection{Comunicación efectiva}
Carnegie enfatiza la importancia de escuchar activamente y comprender las necesidades de los demás. En un entorno organizacional, esto se traduce en una comunicación transparente y abierta entre colegas y líderes.

\subsubsection{Relaciones positivas}
El libro promueve la construcción de relaciones sólidas basadas en la empatía y el reconocimiento de logros. Estos principios son fundamentales para una cultura organizacional saludable.

\subsection{Ambiente de trabajo}
Se refiere al entorno físico y psicológico en el que los empleados realizan sus tareas. El ambiente de trabajo incluye factores como la cultura laboral, las relaciones interpersonales, la seguridad, la ergonomía, la distribución del espacio, la ilumniación, la temperatura, entre otros.

Por lo tanto, un buen ambiente de trabajo promueve la productividad, el bienestar y la satisfacción de los empleados. De acuerdo a Daniel Goleman en la dirección de personal se necesita inteligencia emocional para tener la capacidad de reconocer y gestionar las emociones propias y ajenas.

\subsubsection{Autoconciencia y autogestión}
La inteligencia emocional ayuda a los empleados y líderes a comprender sus propias emociones y manejar el estrés. Esto contribuye a un ambiente de trabajo menos tenso y más productivo.

\subsubsection{Empatía y habilidades sociales}
La empatía hacia los colegas y la habilidad para trabajar en equipo son esenciales para crear un ambiente laboral positivo.

\section{Conclusion}

\section{Recomendaciones}

\appendices
\section{Proof of the First Zonklar Equation}
Appendix one text goes here.

\section{}
Appendix two text goes here.

\section*{Acknowledgment}

The authors would like to thank...

\begin{thebibliography}{1}
  
\bibitem{carnegie1936}
D.~Carnegie, ``Cómo ganar amigos e influir sobre las personas,'' \emph{Simon and Schuster}, 1936.
  
\bibitem{chavez2024}
D.~Chavez, ``Comunicación efectiva en la gestión de proyectos,'' \emph{LinkedIn}, 2024. [En línea]. Disponible: \url{https://www.linkedin.com/pulse/comunicaci%C3%B3n-efectiva-en-la-gesti%C3%B3n-de-proyectos-diego-chavez-1btqf/?originalSubdomain=es}. [Accedido: 20-Feb-2024].

\bibitem{questionpro}
C.~Ortega,  ``Cultura Organizacional'' \emph{QuestionPro}, 2024. [En línea]. Disponible: \url{https://www.questionpro.com/blog/es/cultura-organizacional-2/}. [Accedido: 29-Feb-2024].

\end{thebibliography}

\end{document}


