\documentclass[journal]{IEEEtran}
\usepackage[utf8]{inputenc}
\usepackage[T1]{fontenc}
\usepackage[spanish]{babel}
\usepackage{graphicx}
\usepackage{url}
\usepackage{hyperref}
\usepackage{float}
\hyphenation{op-tical net-works semi-conduc-tor}

\begin{document}

% paper title
\title{Dirección Efectiva en la Ingeniería de Sistemas Informáticos: Un Análisis de la Inteligencia Emocional y las Técnicas de Relaciones Humanas}

\author{Elmer Edenilson Rosales Molina,\\ ~\IEEEmembership{rm20001@ues.edu.sv;}\\
        Julio Cesar Mejía Rodriguez,\\ ~\IEEEmembership{mr20084@ues.edu.sv}}

\maketitle

\begin{abstract}
The abstract goes here.
\end{abstract}

\begin{keywords}
IEEEtran, journal, \LaTeX, paper, template.
\end{keywords}

\IEEEpeerreviewmaketitle

\section{Introducción}
Muchas veces pensamos que las personas con las mejores notas o con mucha preparación profesional son los mejores candidatos para el desempeño de puestos de liderazgo en las empresas u organizaciones, pero un aspecto esencial que casi nunca es tomado en cuenta, es la capacidad de inteligencia emocional que los individuos poseen, este tipo de inteligencia abarca 5 habilidades, que según Goleman son susceptibles de aprenderse y perfeccionase a lo largo de la vida, si para ello se utilizan los métodos adecuados.

Estas habilidades son: Autocontrol, Entusiasmo, Empatía, Perseverancia y Automotivación. 

El diseño biológico que rige nuestras emociones constituye un impulso que nos moviliza a la acción, es por ellos que el mismo significado de la palabra en latín “moveré” significa moverse y el prefijo e denota un objetivo.

La emoción, entonces, desde el plano semántico, significa “movimiento hacia”, y basta con observar a los animales o a los niños pequeños para encontrar la forma en que las emociones los dirigen hacia una acción determinada, que puede ser huir, llora, etc.

Nuestra experiencia en la vida y el medio en el cual hayamos vivido moldean nuestras respuestas ante los estímulos emocionales que encontramos.

Alrededor del tallo encefálico, que constituye la región más primitiva de nuestro cerebro y que regula las funciones básicas como la respiración o el metabolismo, se fue configurando el sistema límbico, que aporta las emociones al repertorio de respuestas cerebrales. Gracias a éste, nuestros primeros ancestros pudieron ir ajustando sus acciones para adaptarse a las exigencias de un entorno cambiante.

Así, fueron desarrollando la capacidad de identificar los peligros, temerlos y evitarlos. La evolución del sistema límbico estuvo, por tanto, aparejada al desarrollo de dos potentes herramientas: la memoria y el aprendizaje.

En esta región cerebral se ubica la amígdala, que tiene la forma de una almendra y que, de hecho, recibe su nombre del vocablo griego que denomina a esta última. Se trata de una estructura pequeña, aunque bastante grande en comparación con la de nuestros parientes evolutivos, en la que se depositan nuestros recuerdos emocionales y que, por ello mismo, nos permite otorgarle significado a la vida. Sin ella, nos resultaría imposible reconocer las cosas que ya hemos visto y atribuirles algún valor.

Sobre esta base cerebral en la que se asientan las emociones, fue creándose hace unos cien millones de años el neocórtex: la región cerebral que nos diferencia de todas las demás especies y en la que reposa todo lo característicamente humano. El pensamiento, la reflexión sobre los sentimientos, la comprensión de símbolos, el arte, la cultura y la civilización encuentran su origen en este esponjoso reducto de tejidos neuronales.

\hfill Febrero 02, 2024

\section{Objetivos}

\subsection{Objetivo general}
Evaluar y fortalecer la dirección efectiva en la Ingeniería de Sistemas Informáticos mediante el análisis de la inteligencia emocional y las técnicas de relaciones humanas.

\subsection{Objetivos específicos}
\begin{itemize}
	\item Explorar el grado de conciencia emocional que tienen los estudiantes de ingeniería de sistemas informáticos de la Facultad de Ingeniería y Arquitectura de la Universidad De El Salvador, Cede Central.
	\item Indagar sobre la percepción de los estudiantes que cursan la carrera de Ingeniería de Sistemas Informáticos en la Universidad de El Salvador, acerca de la relevancia de la inteligencia emocional y las relaciones interpersonales en el desarrollo de habilidades profesionales y el éxito académico.
	\item Investigar cómo la inteligencia emocional y la buena comunicación influye en la capacidad de los estudiantes inscritos en la carrera de Ingeniería de Sistemas Informáticos en la Universidad De El Salvador para resolver problemas de manera efectiva y colaborativa en equipos de trabajo.
	\item Analizar los niveles de estrés percibidos por los estudiantes de Ingeniería de Sistemas informáticos de la Universidad De El Salvador y cómo la inteligencia emocional puede influir en su bienestar emocional y manejo del estrés.
	
\end{itemize}

\section{Alcances}

\section{Organización y Relaciones Interpersonales}
\PARstart{E}{n} el mundo de la dirección, especialmente en el campo de la Ingeniería de Sistemas Informáticos, la organización y las relaciones interpersonales asumen un papel crucial. Debido a que la organización hace referencia a cómo se estructuran y coordinan las actividades y los recursos dentro de una empresa (lo cual no es tarea fácil), por consiguiente, las relaciones interpersonales se refieren a cómo los miembros del equipo interactúan entre sí para obtener metas establecidas.

En su libro “Cómo ganar amigos e influir sobre las personas”, Dale Carnegie presenta varias técnicas que pueden mejorar las relaciones interpersonales en un entorno de trabajo. Por ejemplo, sugiere que mostrar un interés genuino en los demás y ser un buen oyente puede ayudar a construir relaciones más fuertes y más positivas. Estas técnicas pueden ser especialmente útiles en un entorno de TI, donde la colaboración y la comunicación efectiva son esenciales para el éxito del proyecto.

Además, Carnegie enfatiza la importancia de hacer que las personas se sientan importantes y valoradas. Esta técnica puede ser especialmente relevante para la organización en un entorno de TI. Al hacer que los miembros del equipo se sientan valorados, los líderes pueden aumentar la motivación y el compromiso del equipo, lo cual puede llevar a una mayor productividad y eficacia.

Por otro lado, en su libro “Inteligencia Emocional”, Daniel Goleman argumenta que la capacidad para reconocer, entender y gestionar nuestras emociones y las de los demás es crucial para las relaciones interpersonales. En un entorno de TI, donde los proyectos a menudo pueden ser estresantes y desafiantes, la inteligencia emocional puede ayudar a los líderes a manejar mejor el estrés y a resolver conflictos de manera efectiva.

%\begin{table}[H]
%% increase table row spacing, adjust to taste
%\renewcommand{\arraystretch}{1.3}
%\caption{An Example of a Table}
%\label{table_example}
%\centering
%\begin{tabular}{|c||c|}
%\hline
%One & Two\\
%\hline
%Three & Four\\
%\hline
%\end{tabular}
%\end{table}

\section{Dirección en el área de informática}
La dirección en el área de informática en las empresas es de vital importancia, ya que estas áreas son las encargadas de velar y administrar por la información crucial y esencial para el cumplimiento de las metas empresariales, es por ello que surge la necesidad de estudiar desde un punto psicoemocional el perfil de la/las personas destinadas a la administración y dirección del área de informática.

Este tópico adquiere una relevancia aún mayor cuando se considera el impacto de la inteligencia emocional en la gestión eficaz de estos departamentos. En su libro "Inteligencia Emocional", Daniel Goleman destaca cómo las habilidades emocionales son fundamentales para liderar equipos en cualquier contexto, y esto se aplica especialmente en el ámbito de la informática.

En un entorno altamente técnico y en constante evolución como es la informática, los líderes deben poseer una comprensión profunda tanto de la tecnología como de las personas que operan dentro de este campo. Es aquí donde entra en juego la inteligencia emocional, que se convierte en un elemento crucial para el éxito de la dirección en el área de informática.

Goleman señala que la inteligencia emocional abarca competencias como la autoconciencia ya que comprender las fortalezas, debilidades, valores y motivaciones personales y grupales permite tomar decisiones más informadas y efectivas. En el ámbito de TI, donde la toma de decisiones puede tener consecuencias significativas, la autoconciencia ayuda a evaluar objetivamente las situaciones y a liderar con claridad y determinación, otra es el autocontrol y la autogestión, estos son aspectos cruciales en el área de TI. En un entorno donde los plazos son ajustados y los desafíos son constantes, la capacidad de manejar el estrés, mantener la calma bajo presión y tomar decisiones racionales es esencial. Ya que la autogestión y autocontrol permite mantener el enfoque en los objetivos, superar los obstáculos y mantener la productividad incluso en las circunstancias más desafiantes, por otro lado, la empatía es también otra habilidad indispensable para la eficiente dirección de las áreas informáticas. Comprender las emociones y perspectivas de los miembros del equipo y de los superiores permite construir relaciones sólidas, fomentar un ambiente de trabajo positivo y promover la colaboración y el trabajo en equipo. Además, la empatía es una gran herramienta para abordar los conflictos de manera constructiva y también para motivar a nuestro equipo de manera efectiva hacia el logro de los objetivos comunes, y, por último, pero no menos importante se tienen a las habilidades sociales, estas desempeñan un papel crucial en el éxito de los líderes informáticos. La capacidad de comunicarse claramente, escuchar activamente y proporcionar retroalimentación constructiva es fundamental para dirigir un equipo de manera efectiva. Ya que las habilidades sociales permiten inspirar confianza, fomentar la creatividad y construir relaciones sólidas tanto dentro como fuera del equipo, lo que es esencial en un campo tan colaborativo como el de la informática.

La dirección en el área de informática va más allá de simplemente gestionar tecnología y datos; implica también entender y guiar a las personas que conforman estos equipos. La inteligencia emocional, tal como describe Goleman, emerge como un componente esencial en este proceso. Desde la autoconciencia hasta las habilidades sociales, cada aspecto de la inteligencia emocional expresada por Goleman se convierte en una herramienta poderosa para los líderes informáticos en la búsqueda del éxito organizacional. Al integrar estos principios en la dirección de áreas de TI, las empresas pueden cultivar entornos de trabajo más colaborativos, productivos y adaptativos, impulsando así su competitividad en un mundo empresarial cada vez más digitalizado y dinámico.

\section{Comunicación Efectiva en la Dirección de Proyectos de TI}
La comunicación efectiva se considera una habilidad esencial para el éxito de cualquier proyecto, debido a que todos los participantes muestran lo mejor de sí mismos y actuan con sinergia en todas las actvidades del proyecto al crear un clima de trabajo agradable. 

Al hacer referencia en el campo de la Ingeniería de Sistemas Informáticos los proyectos de TI suelen ser complejos y requieren de un equipo compuesto por individuos con diferentes habilidades y conocimientos. Usualmente estos son multidisciplinarios y dinámicos, se necesita interactuar constantemente con diferentes actores.

Según Chavez, “Una buena comunicación permite establecer objetivos claros, alinear expectativas, resolver conflictos, compartir información, obtener feedback, generar confianza y fomentar la colaboración. Todo esto contribuye a mejorar la calidad, la eficiencia y la satisfacción de los proyectos de TI”.

\subsection{Aspectos que deben considerarse en cada proyecto}
En proyectos informáticos, la calidad del producto se debe múltiples factores que pueden ser cumplidos al trabajar en conjunto y de forma coordinada, por esta razón, el dirigente del proyecto debe considerar los siguientes puntos:
\begin{itemize}
	\item Definir el propósito y el alcance del proyecto, así como los roles y responsabilidades de cada participante.
	\item Utilizar los canales y herramientas adecuados para cada tipo de comunicación, ya sea oral, escrita, sincrónica o asincrónica.
	\item Ser claro, conciso, coherente y cortés en tus mensajes, evitando ambigüedades, jergas y tecnicismos innecesarios.
	\item Escuchar activamente a los demás, mostrando empatía, respeto y apertura.
	\item Solicitar y ofrece feedback constructivo, reconociendo los logros y las oportunidades de mejora.
	\item Mantener una comunicación fluida y frecuente, informando sobre el estado, los avances y los problemas del proyecto.
	\item Fomentar la participación y la colaboración de todos los involucrados, creando un ambiente de confianza y transparencia.
\end{itemize}

\subsection{El enfoque de Carnegie para mejorar la comunicación}
Daniel Carnegie en su libro “Cómo ganar amigos e influir sobre las personas” proporciona técnicas para interactuar eficazmente con las personas lo cual contribuye a la buena comunicación entre los miembros del equipo. Por lo tanto, se citarán algunas de estas tecnicas:

\subsubsection{No critique, no condene ni se queje}
Según Carnegie, “La crítica es inútil porque pone a la otra persona en la defensiva, y por lo común hace que trate de justificarse. La crítica es peligrosa porque lastima el orgullo, tan precioso de la persona, hiere su sentido de la importancia y despierta su resentimiento.”

Ahora bien, de acuerdo a lo expuesto por este autor es posible que el líder de un proyecto se sienta tentado ha actuar con esa conducta lo cual no ayudaría en mucho. Si se analiza detenidamente en escenas de trabajo, podemos obtener lo siguiente:

\begin{itemize}
	\item \textbf{No critique:} En la dirección de proyectos de TI, es común encontrar errores y problemas. Sin embargo, criticar a los miembros del equipo por sus errores puede crear un ambiente de trabajo negativo y disminuir la moral del equipo. En lugar de criticar, un líder efectivo proporciona retroalimentación constructiva y ayuda a los miembros del equipo a aprender de sus errores.
	\item \textbf{No condene:} Condenar a los miembros del equipo por sus acciones puede crear resentimiento y conflictos dentro del equipo. Un líder efectivo se enfoca en resolver el problema en lugar de condenar a la persona. Esto puede incluir discutir el problema de manera abierta y honesta, buscar soluciones juntos y aprender de la experiencia.
	\item \textbf{No se queje: }Quejarse puede ser perjudicial para el ambiente de trabajo y puede disminuir la productividad del equipo. En lugar de quejarse, un líder efectivo se enfoca en encontrar soluciones a los problemas y mantener una actitud positiva.
\end{itemize}

Por lo expuesto anteriormente, metodologías agiles como SCRUM ofrecen espacios entre los miembros para exponer los avances realizados, además, cuando uno de ellos presenta dificultades para desempeñar su tarea todos los miembros del equipo unen su conocimiento y experiencia para solucionar el problema lo cual hace pensar la importancia de la comunicación.

\subsubsection{Demuestre aprecio honrado y sincero}
Se cita un viejo dicho, “Pasaré una sola vez por este camino; de modo que cualquier bien que pueda hacer o cualquier cortesía que pueda tener para con cualquier ser humano, que sea ahora. No la dejaré para mañana, ni la olvidaré, porque nunca más volveré a pasar por aquí.”

Según Carnegie, “Herir a la gente no sólo no la cambia, sino que es una tarea que nadie nos agradecerá”. En esta regla señala que el caracter humano posee una principio profundo que es el anhelo de ser apreciado, eso emmarca algo significativo en las personas y satisfacen sus deseos lo cual es muy diferente a la adulación porque genera sentido de importancia y mejora la calidad.

En ingeniería se puede reflexionar en la siguiente aplicación:
\begin{itemize}
	\item \textbf{Reconocimiento del trabajo bien hecho:} En la dirección de proyectos de TI, es importante reconocer y apreciar el trabajo bien hecho. Esto puede ser tan simple como agradecer a un miembro del equipo por su contribución o destacar el buen trabajo en una reunión de equipo. Este tipo de aprecio sincero puede aumentar la moral del equipo y motivar a los miembros del equipo a continuar haciendo un buen trabajo.
	\item \textbf{Valoración de las ideas y sugerencias:} Los líderes de proyectos de TI deben demostrar aprecio por las ideas y sugerencias de los miembros del equipo. Esto no solo fomenta un ambiente de innovación y creatividad, sino que también hace que los miembros del equipo se sientan valorados y respetados.
	\item \textbf{Apoyo durante los desafíos:} Los proyectos de TI a menudo pueden ser desafiantes y estresantes. Durante estos tiempos, demostrar aprecio por el esfuerzo y la dedicación de los miembros del equipo puede proporcionar un muy necesario impulso moral.
\end{itemize}

\subsubsection{Despierte en los demás un deseo vehemente}
En la dirección de proyectos de TI, es importante inspirar a los miembros del equipo a alcanzar sus metas y objetivos. Según Carnegie, “el único medio que se tiene para influir sobre la otra persona es hablar acerca de lo que se quiere y demostrarle cómo conseguirlo”. Esto puede implicar ayudar a los miembros del equipo a ver cómo su trabajo contribuye al éxito general del proyecto y cómo pueden crecer y desarrollarse profesionalmente.

Además, despertar un deseo vehemente en los demás también puede ser una poderosa herramienta de motivación. Carnegie argumenta que “quien puede hacer las cosas tiene al mundo entero consigo. Quien no puede, marcha solo por el camino”. En un entorno de TI, esto podría implicar motivar a los miembros del equipo a aprender nuevas habilidades, adoptar nuevas tecnologías o asumir nuevos desafíos.

Al despertar un deseo vehemente en los demás, los líderes de proyectos de TI pueden influir positivamente en sus equipos. Esto puede llevar a un mayor compromiso, una mayor productividad y, en última instancia, a un mayor éxito del proyecto.

\subsubsection{Seis maneras de agradar a los demás}
\begin{itemize}
	\item \textbf{Interésese sinceramente por los demás:} Carnegie sugiere mostrar un interés genuino en los demás. En la dirección de proyectos de TI, esto puede traducirse en prestar atención a las ideas y opiniones de los miembros del equipo, lo cual puede fomentar un ambiente de trabajo colaborativo y respetuoso.
	\item \textbf{Sonría:} Una actitud positiva puede ser contagiosa y puede ayudar a crear un ambiente de trabajo más agradable y productivo.
	\item \textbf{Recuerde que para toda persona su nombre es el sonido más dulce e importante en cualquier idioma:} Recordar y usar los nombres de los miembros del equipo puede demostrar respeto y aprecio, lo cual puede mejorar las relaciones interpersonales y la comunicación.
	\item \textbf{Sea un buen oyente. Anime a los demás a hablar de sí mismos:} Escuchar activamente a los miembros del equipo puede demostrar que valoras sus ideas y opiniones, lo cual puede fomentar una comunicación abierta y efectiva.
	\item \textbf{Hable en términos de los intereses de la otra persona:} Comprender y hablar en términos de los intereses de los miembros del equipo puede ayudar a motivarlos y a obtener su cooperación.
	\item \textbf{Haga que la otra persona se sienta importante y hágalo sinceramente:} Reconocer el valor y las contribuciones de los miembros del equipo puede aumentar su moral y su compromiso con el proyecto.
\end{itemize}


\section{Gestión de Conflictos en Proyectos Informáticos}
En la gestión de proyectos, especialmente en entornos dinámicos y colaborativos como el de TI los conflictos son inevitables. Estos pueden surgir debido a diferencias de opiniones, objetivos divergentes o simplemente por la naturaleza compleja y multifacética de los proyectos. Es por esto que entre las habilidades que destaca Goleman en su libro de inteligencia emocional, la empatía que proviene del griego empatheia (sentir dentro), emerge como una habilidad fundamental para abordar los conflictos de manera efectiva y construir relaciones sólidas entre los miembros del equipo, ya que denota la capacidad de percibir la experiencia subjetiva de otra persona.

Cuando surge un conflicto en un proyecto, las personas involucradas pueden tener diferentes puntos de vista, intereses y preocupaciones. La empatía permite que los líderes y miembros del equipo a que reconozcan y comprendan estas diferencias, lo que facilita la comunicación abierta y la búsqueda de soluciones mutuamente beneficiosas. Al ponerse en el lugar de los demás, los líderes y miembros del equipo desarrollan una comprensión más profunda de las causas subyacentes del conflicto y trabajan hacia soluciones que aborden estas preocupaciones de manera efectiva.

También, la empatía facilita la comunicación abierta y honesta entre las partes en conflicto. Al demostrar comprensión y respeto por las perspectivas de los demás, se crea un ambiente de confianza que fomenta la expresión libre de ideas y opiniones. Esto permite que todas las partes involucradas en el conflicto se comuniquen de manera más efectiva y colaboren en la búsqueda de soluciones mutuamente satisfactorias.

La empatía también contribuye a construir relaciones sólidas y de confianza entre los miembros del equipo, lo que es fundamental para la resolución de conflictos a largo plazo. Al demostrar empatía hacia los demás, se fortalecen los lazos de equipo y se crea un sentido de pertenencia y amistad que puede ayudar a superar las diferencias y obstáculos que surjan durante el proyecto.

La empatía no solo ayuda a resolver conflictos de manera efectiva, sino que también fortalece la cohesión del equipo y promueve un ambiente de trabajo colaborativo y productivo. Al cultivar la empatía en la gestión de proyectos informáticos, los líderes y equipos pueden superar los desafíos con mayor eficacia y alcanzar el éxito del proyecto de manera más consistente.

Además, Goleman argumenta que, en los conflictos, es crucial que predominen la mente emocionalmente inteligente sobre la mente impulsiva o reactiva. Goleman sugiere que una persona emocionalmente inteligente es capaz de manejar eficazmente las emociones propias y las de los demás en situaciones conflictivas.

La mente emocionalmente inteligente se caracteriza por la autoconciencia, la autorregulación, la empatía y las habilidades sociales. Estas habilidades permiten a una persona entender sus propias emociones y las de los demás, controlar sus impulsos y reacciones emocionales, mostrar empatía hacia los puntos de vista de los demás y comunicarse de manera efectiva para resolver conflictos de manera constructiva.

\section{Motivación y Compromiso del Equipo}
La sinergia que se genera cuando un grupo de individuos comparte objetivos comunes y se compromete con su consecución es un componente vital en cualquier entorno colaborativo. En este sentido, la motivación y el compromiso del equipo se rigen como pilares fundamentales que sustentan el éxito y la efectividad de cualquier empresa u organización. Explorar el porqué de su importancia no solo lleva a comprender cómo se forja un equipo exitoso, sino que también sumerge en la dinámica de las relaciones humanas y la gestión del talento.

En el contexto de un equipo, la motivación se refiere al impulso o la fuerza interna que lleva a los miembros del equipo a actuar de cierta manera, perseguir metas específicas o completar tareas asignadas. La motivación puede surgir de diversas fuentes, como la búsqueda de logros personales, el deseo de contribuir al éxito del equipo, el reconocimiento de los compañeros o líderes, o la satisfacción derivada del trabajo bien hecho. La motivación es fundamental para mantener altos niveles de compromiso y productividad en el equipo, ya que ayuda a los miembros a superar obstáculos, mantenerse enfocados en los objetivos y perseverar ante los desafíos.

Por otro lado, el compromiso se refiere a la dedicación y la lealtad de los miembros hacia los objetivos comunes del equipo y hacia el éxito colectivo. Implica un sentido de responsabilidad compartida, donde cada miembro está dispuesto a contribuir con su tiempo, energía y habilidades para alcanzar las metas del equipo. El compromiso también implica una conexión emocional con el equipo y sus objetivos, lo que lleva a una mayor motivación para trabajar juntos, colaborar eficazmente y superar cualquier adversidad que pueda surgir en el camino hacia el éxito del equipo. 

Esto conlleva a destacar la importancia de la inteligencia emocional, ya que esta permite una serie de capacidades y habilidades que son fundamentales tanto en el ámbito personal como profesional.

En un entorno de equipo, la inteligencia emocional se manifiesta en la capacidad de los miembros del equipo para comprender y gestionar tanto sus propias emociones como las de los demás. Esta habilidad no solo promueve una mayor armonía y colaboración dentro del equipo, sino que también es fundamental para fomentar un compromiso colectivo significativo.

Los equipos con una alta inteligencia emocional pueden identificar y comprender las dinámicas emocionales dentro del grupo. Esto incluye la capacidad de reconocer las motivaciones individuales de cada miembro, así como sus fortalezas y debilidades emocionales. Al tener una comprensión clara de las necesidades y metas de cada miembro, el equipo puede comprometerse con proyectos y actividades que estén alineados con los intereses y valores compartidos, lo que aumenta el nivel de dedicación colectiva.

Además, un equipo con una buena inteligencia emocional es capaz de mantener la calma y la compostura incluso en situaciones desafiantes. Esto permite que el equipo se mantenga enfocado en sus responsabilidades y metas, incluso cuando enfrentan obstáculos y dificultades. La capacidad de regular eficazmente las emociones y mantener una actitud positiva contribuye en gran medida a mantener un alto nivel de compromiso y rendimiento en el equipo.

La importancia de que los equipos tenga motivación y compromiso es que estos tienden a trabajar con mayor eficiencia y dedicación, por ende este comportamiento conduce al aumento de la productividad del equipo, en el desarrollo de las tareas y actividades a cargo, además el ambiente de trabajo se torna ameno, ya que los miembros del equipo se inclinan a trabajar juntos, compartir ideas sin ningún temor y también para lograr las metas comunes, por ultimo estas características los convierte en un equipo resiliente frente a desafíos y adversidades que pueden surgir en el camino.


\section{El Liderazgo}
En la dirección de proyectos de TI, es común encontrarse con errores y problemas complejos. Sin embargo, la forma en que un líder maneja estos problemas puede tener un gran impacto en el equipo y en el éxito del proyecto. Dale Carnegie ofrece varias técnicas que pueden ayudar a los líderes a manejar estos problemas de manera efectiva.

\subsection{Enfoque de Dale Carnegie}

\subsubsection{Elogios y aprecio sincero}
Para abordar un problema o un error primeramente Carnegie sugiere comenzar con elogios y aprecio sincero con el fin de obtener respuestas positivas. Esto puede ayudar a suavizar el golpe de la crítica y a hacer que la otra persona esté más abierta a la retroalimentación.

\subsubsection{Llamar la atención indirectamente}
En lugar de señalar directamente los errores de los demás, Carnegie sugiere hacerlo de manera indirecta. Esto puede ser especialmente útil en un entorno de TI, donde la crítica directa puede ser mal recibida y causar resentimiento.
 
\subsubsection{Hablar primero de nuestros errores antes de críticar a los demás}
Admitir nuestros propios errores antes de señalar los de los demás puede hacer que la crítica sea más fácil de aceptar. Esto puede ser especialmente relevante en un entorno de TI, donde los errores son a menudo inevitables y la mejora continua es clave.
 
\subsubsection{Hacer preguntas en lugar de dar órdenes directas}
Carnegie sugiere que hacer preguntas puede ser una forma efectiva de influir en los demás sin causar resentimiento. En un entorno de TI, esto podría implicar pedir a los miembros del equipo su opinión o sugerencias sobre cómo resolver un problema en lugar de simplemente decirles qué hacer.
 
\subsection{Enfoque de Daniel Goleman}
 
\subsubsection{Autoconsciencia:} Goleman argumenta que la autoconsciencia, la capacidad de reconocer y entender nuestras propias emociones, es la base de la inteligencia emocional. En la dirección de proyectos de TI, la autoconsciencia puede ayudar a los líderes a entender cómo sus emociones pueden influir en su comportamiento y en sus decisiones.
\subsubsection{Autorregulación:} La autorregulación, la capacidad de manejar nuestras emociones de manera efectiva, es otro aspecto clave de la inteligencia emocional. Los líderes que pueden manejar sus emociones son menos propensos a tomar decisiones impulsivas o a reaccionar de manera exagerada ante los problemas.
\subsubsection{Empatía:} La empatía, la capacidad de entender y compartir los sentimientos de los demás, es crucial para el liderazgo efectivo. Los líderes empáticos pueden entender mejor las necesidades y preocupaciones de sus equipos, lo cual puede mejorar la comunicación y la colaboración.
\subsubsection{Habilidades sociales:} Goleman también destaca la importancia de las habilidades sociales, como la capacidad de manejar las relaciones y navegar por las redes sociales. En un entorno de TI, donde la colaboración es clave, las habilidades sociales pueden ayudar a los líderes a construir relaciones fuertes y productivas con sus equipos.

\section{La Cultura Organizacional y Ambiente de Trabajo}
Son aspectos fundamentales dentro de una organización porque expresan lo bien de una entidad con el trato al personal de trabajo, y por ende, las buenas ideas le dan lugar al negocio.

\subsection{Cultura organizacional}
La cultura organizacional es el conjunto de valores, creencias, normas, costumbres y comportamientos compartidos por los miembros de una organización.

Define la identidad de la empresa y afecta cómo las personas interactuan entre sí y con la organización en su conjunto. Influye en la toma de decisiones, la comunicación, la motivación y la forma en que abordan los desafíos.

Dale Carnegie, en su libro "Cómo ganar amigos e influir sobre las personas" se centra exclusivamente en las habilidades interpersonales y de comunicación, pues está interesado en mostrar una perspectiva motivacional en la interacción con las personas. De acuerdo al autor podemos mostrar dos aspectos relevantes ha considerar para la cultura organizacional:

\subsubsection{Comunicación efectiva}
Carnegie enfatiza la importancia de escuchar activamente y comprender las necesidades de los demás. En un entorno organizacional, esto se traduce en una comunicación transparente y abierta entre colegas y líderes.

\subsubsection{Relaciones positivas}
El libro promueve la construcción de relaciones sólidas basadas en la empatía y el reconocimiento de logros. Estos principios son fundamentales para una cultura organizacional saludable.

\subsection{Ambiente de trabajo}
Se refiere al entorno físico y psicológico en el que los empleados realizan sus tareas. El ambiente de trabajo incluye factores como la cultura laboral, las relaciones interpersonales, la seguridad, la ergonomía, la distribución del espacio, la ilumniación, la temperatura, entre otros.

Por lo tanto, un buen ambiente de trabajo promueve la productividad, el bienestar y la satisfacción de los empleados. De acuerdo a Daniel Goleman en la dirección de personal se necesita inteligencia emocional para tener la capacidad de reconocer y gestionar las emociones propias y ajenas.

\subsubsection{Autoconciencia y autogestión}
La inteligencia emocional ayuda a los empleados y líderes a comprender sus propias emociones y manejar el estrés. Esto contribuye a un ambiente de trabajo menos tenso y más productivo.

\subsubsection{Empatía y habilidades sociales}
La empatía hacia los colegas y la habilidad para trabajar en equipo son esenciales para crear un ambiente laboral positivo.

\section{Metodología de la Investigación}
\subsection{Diseño de la investigación}
La presente investigación enmarca un estudio de naturaleza cuantitativa. Este enfoque se ha seleccionado con el propósito de cuantificar y analizar de manera sistemática las respuestas de los participantes en la encuesta, permitiendo así identificar patrones, tendencias y relaciones entre variables relacionadas con la inteligencia emocional y la interacción personal en estudiantes que cursan la carrera de Ingeniería de Sistemas Informáticos en la Universidad de El Salvador. A través de la recopilación de datos numéricos, se buscará comprender en profundidad el grado de conciencia emocional de los estudiantes, las técnicas que utilizan para mejorar las relaciones interpersonales, su percepción sobre la relevancia de la inteligencia emocional en el ámbito académico y profesional, así como su influencia en la resolución de problemas y el manejo del estrés. El enfoque cuantitativo proporcionará una base sólida para el análisis estadístico de los datos, permitiendo la identificación de tendencias significativas y la formulación de conclusiones respaldadas por evidencia empírica.
\subsection{Diseño de la encuesta}

\section{Conclusion}

\section{Recomendaciones}

\appendices
\section{Proof of the First Zonklar Equation}
Appendix one text goes here.

\section{}
Appendix two text goes here.

\section*{Acknowledgment}

The authors would like to thank...

\begin{thebibliography}{1}
  
\bibitem{carnegie1936}
D.~Carnegie, ``Cómo ganar amigos e influir sobre las personas,'' \emph{Simon and Schuster}, 1936.
  
\bibitem{chavez2024}
D.~Chavez, ``Comunicación efectiva en la gestión de proyectos,'' \emph{LinkedIn}, 2024. [En línea]. Disponible: \url{https://www.linkedin.com/pulse/comunicaci%C3%B3n-efectiva-en-la-gesti%C3%B3n-de-proyectos-diego-chavez-1btqf/?originalSubdomain=es}. [Accedido: 20-Feb-2024].

\bibitem{questionpro}
C.~Ortega,  ``Cultura Organizacional'' \emph{QuestionPro}, 2024. [En línea]. Disponible: \url{https://www.questionpro.com/blog/es/cultura-organizacional-2/}. [Accedido: 29-Feb-2024].

\end{thebibliography}

\end{document}


