\documentclass[journal]{IEEEtran}
\usepackage[utf8]{inputenc}
\usepackage[T1]{fontenc}
\usepackage[spanish]{babel}
\usepackage{graphicx}
\usepackage{url}
\usepackage{hyperref}
\usepackage{float}
\hyphenation{op-tical net-works semi-conduc-tor}

\begin{document}

% paper title
\title{Dirección Efectiva en la Ingeniería de Sistemas Informáticos: Un Análisis de la Inteligencia Emocional y las Técnicas de Relaciones Humanas}

\author{Elmer Edenilson Rosales Molina,\\ ~\IEEEmembership{rm20001@ues.edu.sv;}\\
        Julio Cesar Mejía Rodriguez,\\ ~\IEEEmembership{mr20084@ues.edu.sv}}

\maketitle

\begin{abstract}
The abstract goes here.
\end{abstract}

\begin{keywords}
IEEEtran, journal, \LaTeX, paper, template.
\end{keywords}

\IEEEpeerreviewmaketitle

\section{Introducción}

 
\hfill Febrero 02, 2024

\section{Objetivos}

\section{Alcances}

\section{Organización y Relaciones Interpersonales}
\PARstart{E}{n} el mundo de la dirección, especialmente en el campo de la Ingeniería de Sistemas Informáticos, la organización y las relaciones interpersonales asumen un papel crucial. Debido a que la organización hace referencia a cómo se estructuran y coordinan las actividades y los recursos dentro de una empresa (lo cual no es tarea fácil), por consiguiente, las relaciones interpersonales se refieren a cómo los miembros del equipo interactúan entre sí para obtener metas establecidas.

En su libro “Cómo ganar amigos e influir sobre las personas”, Dale Carnegie presenta varias técnicas que pueden mejorar las relaciones interpersonales en un entorno de trabajo. Por ejemplo, sugiere que mostrar un interés genuino en los demás y ser un buen oyente puede ayudar a construir relaciones más fuertes y más positivas. Estas técnicas pueden ser especialmente útiles en un entorno de TI, donde la colaboración y la comunicación efectiva son esenciales para el éxito del proyecto.

Además, Carnegie enfatiza la importancia de hacer que las personas se sientan importantes y valoradas. Esta técnica puede ser especialmente relevante para la organización en un entorno de TI. Al hacer que los miembros del equipo se sientan valorados, los líderes pueden aumentar la motivación y el compromiso del equipo, lo cual puede llevar a una mayor productividad y eficacia.

Por otro lado, en su libro “Inteligencia Emocional”, Daniel Goleman argumenta que la capacidad para reconocer, entender y gestionar nuestras emociones y las de los demás es crucial para las relaciones interpersonales. En un entorno de TI, donde los proyectos a menudo pueden ser estresantes y desafiantes, la inteligencia emocional puede ayudar a los líderes a manejar mejor el estrés y a resolver conflictos de manera efectiva

\subsection{Subsection Heading Here}
Subsection text here.

\subsubsection{Subsubsection Heading Here}
Subsubsection text here.


%\begin{table}[H]
%% increase table row spacing, adjust to taste
%\renewcommand{\arraystretch}{1.3}
%\caption{An Example of a Table}
%\label{table_example}
%\centering
%\begin{tabular}{|c||c|}
%\hline
%One & Two\\
%\hline
%Three & Four\\
%\hline
%\end{tabular}
%\end{table}


\section{Conclusion}

\section{Recomendaciones}

\appendices
\section{Proof of the First Zonklar Equation}
Appendix one text goes here.

\section{}
Appendix two text goes here.

\section*{Acknowledgment}

The authors would like to thank...

\begin{thebibliography}{1}

\bibitem{IEEEhowto:kopka}
H.~Kopka and P.~W. Daly, \emph{A Guide to {\LaTeX}}, 3rd~ed.\hskip 1em plus
  0.5em minus 0.4em\relax Harlow, England: Addison-Wesley, 1999.

\end{thebibliography}

\end{document}


